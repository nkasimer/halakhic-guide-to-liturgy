\documentclass[11pt]{article}
\usepackage[paperheight=8.5in,paperwidth=5.5in,top=.7in,bottom=.5in, inner=.875in, outer=.75in, marginparsep=.1in, headsep=16pt]{geometry}

\usepackage{polyglossia}
\usepackage{hyperref}
\usepackage{multicol}
\setdefaultlanguage{english}
\setotherlanguage{hebrew}
\RequirePackage{fontspec}

\setmainfont{Times New Roman}

\newfontfamily\hebrewfont{Shlomo}

\newcommand{\hebword}[1]{‎\begin{hebrew}\beginR #1 \endR\end{hebrew}}

\usepackage[parfill]{parskip}
\begin{document}
	
\title{A Brief Guide to the Laws of Prayer}

\author{Compiled by Nathan Kasimer}

\date{}

\maketitle


\section{Introduction}

This guide is intended as a quick-reference for liturgical questions that come up frequently and require an immediate answer, and are more in-depth than siddur instructions typically have. It therefore omits discussion of the obligation of prayer, times for the different liturgies, attire for prayer, and other more regularly-occurring matters of ritual law.  It also omits significant amounts of halakhic discussion and relevant details for the sake of conciseness.

It is primarily based on the text of the English translation of Peninei Halakha, used under the terms of its Creative Commons license, with some additional notes.  For ease of reference, citations will be made both to Peninei Halakha and the sources it cites.  "PH" is used to cite the Peninei Halakha.  Citations to the Shul\d{h}an Arukh are abbreviated to the section of the code ("OC" or "YD") and only the Siman is listed.  This is to allow cross reference to the halakhic works that use the Tur/Shul\d{h}an Arukh's order of Simanim while using a minimum of space.  For laws of Torah Reading, R Karl Applbaum's "Laws Governing the Reading of the Torah, Prophets, and Megillos Esther in the Synagogue" (cited as "Applbaum").

\section{Situations where Prayer is Forbidden}

It is forbidden to pray near uncovered excrement, smelly garbage, or other foul smells.  A person must be four cubits (approximately 2 meters or 6$\frac{1}{2}$ feet) from the source of the smell or where they cannot smell it, whichever is further, even if they themselves cannot smell it (PH Prayer 3:9 citing OC 79).

The above does not apply to feces of babies who do not eat much solid food, since it does not smell as bad as that of adults or animals (PH Prayer 4:3 citing OC 81).

Diapers, catheters, and the like are considered "covered" (PH Prayer 4:4).

One may not hold an item they are afraid will fall during prayer (PH Prayer 5:7 citing OC 96), unless they would be afraid they would be stolen and that would interfere with proper attention to praying (PH Prayer 5:7).

One may not pray when they need to use the toilet (PH Prayer 5:8 citing Berakhot 15a).

A person who is tipsy should not pray, but if they do so their obligation is fulfilled.  A person who is inebriated may not pray (PH Prayer 5:11 citing OC 99).

\section{A Diminished Minyan}

If a few leave during the recital of Kaddish the Kaddish may be completed as long as most of the minyan remains (at least six including the chazan). This rule applies to all prayers that require a minyan (PH Prayer 2:10 citing OC 55).

If a minyan began the repetition of the amidah, and part of the minyan left but most remained, those remaining may say Kedushah and conclude the repetition of the Amidah. However, they do not say Birkat Kohanim because it is a mitzvah by itself. Ashkenazi practice is to continue public liturgy up to and including the following Full Kaddish, but not beyond (ibid). 

\section{Interruptions in Liturgy}

\subsection{During Pesukei DeZimra}

Pesukei DeZimra is the portion of the service between Barukh She'amar and yishtaba\d{h}. Needless interruptions are forbidden during that period, with the following exceptions (based on Koren "Table of Permitted Responses" and PH Prayer 16:5):
\begin{itemize}
	\item Answering "amen" to a blessing 
	\item The congregational response to kedusha and barekhu
	\item Reciting modim derabbanan
	\item Answering \hebword{אמן יהי שמה רבא וכו׳} in kaddish
	\item Reciting the first verse of shema\space with the congregation
	\item Reciting the blessing after using the toilet, on thunder, or on lightening
	\item Respond to a respected person's greeting, or greet a person out of fear
	\item Between paragraphs: greet a respected person, or respond to anyone's greeting
	\item Receive an aliya (though such a person shouldn't be given an aliya unless they're the only Kohen or Levi)
\end{itemize}

In all these cases it is preferable to interrupt between paragraphs if possible, and the interruption should be between verses.

Between Pesukei DeZimra and Sha\d{h}arit it is permitted to make communal announcements that cannot wait until after Full Kaddish (PH Prayer 16:2).  Otherwise, any interruption at this point is forbidden, besides those permitted during Pesukei DeZimra.

\subsection{During the Shema and its Blessings}

During the Shema and its blessings, interruptions are forbidden with the following exceptions (based on Koren "Table of Permitted Responses" and PH Prayer 16:5):
\begin{itemize}
	\item The congregational response to kedusha (but only the lines beginning \hebword{קדוש} and \hebword{ברוך}) and barekhu
	\item Reciting only the words \hebword{מודים אנחנו לך} modim derabbanan
	\item Answering \hebword{אמן יהי שמה רבא וכו׳} in kaddish
	\item Receive an aliya (but not in the middle of the first verse of the Shema)
	\item Respond to a respected person's greeting, or greet a person out of fear
	\item Between paragraphs, greet a respected person, or respond to anyone's greeting
	\item Between paragraphs, recite the blessing on thunder and lighting
	\item Between paragraphs, answer \hebword{אמן} to a blessing
\end{itemize}

\subsection{During the Amidah}

Interruptions during the Amidah are forbidden, even congregational responses to Kaddish and Kedusha. It is considered an interruption to put on a tallit if it has fallen off (though it may be adjusted, OC 97:4).  It is permitted to walk to somewhere else if there is a distraction preventing prayer in the original location (PH Prayer 17:15).

If an interruption is long enough that one could have finished the Amidah in the length of the interruption, the Amidah must be started again (PH Prayer 18:1).

\section{Erring in Seasonal Liturgy}

\subsection{Requests for Rain}

\paragraph{Mashiv HaRua\d{h}}

Mashiv haRua\d{h} is recited beginning at Musaf on the first day of Pesa\d{h} until Musaf on Shemini 'atzeret.  If it is omitted in winter or recited in summer, one must begin the Amidah again (unless the berakha has not been completed, in which case resume from the point of the error).  The only exception is if "Morid HaTal" was recited in summer, in which case the Amidah is continued as normal (PH Prayer 18:4-5 citing OC 114).

\paragraph{Tal uMatar}

The request for rain in the weekday Amidah is recited during the rainy season.  In Israel it is begun on the 7th of Mar\d{h}eshvan at Arvit (i.e. the beginning of the 7th) (PH Prayer 18:5 citing OC 117).

Elsewhere it is begun 60 days after the Autumnal Equinox, which for this purpose is reckoned according to the Julian calendar (tekufat Shemuel).  In the 21st century this works out to Arvit on December 4th, or December 5th in years immidiately before a civil leap year (ibid).  It moves one day on the Gregorian calendar later every time there is a Julian leap year that is not a Gregorian leap year.  This occurs every year that is divisible by 100 but not by 400. When such a year occurs, treat the year before as the year before a leap year.  Then in the non-leap year the request for rain moves one year later.  For example, in the year 2099 the request will begin December 5th, and on December 5th (or 6th, in years before a leap year) thereafter until 2199.

In both locations it is said until Passover. If it was forgotten and the blessing was concluded, it should be inserted in the blessing of \hebword{קולינו שמע} (before \hebword{עננו} if on a fast). If that blessing was concluded, return to the blessing of "mevarekh hashanim".  If the Amidah was completed, it must be recited again (ibid).

\subsection{Holiday Liturgical Additions}

Havdalah in the Amidah is recited at the end of Shabbat and Festivals.  If forgotten the Amidah is not repeated, since Havdalah will also be said separately (PH Prayer 18:2 citing OC 422)

Ya'aleh Ya'avo is recited on festivals and Rosh \d{H}odesh.  If omitted, the Amidah must be repeated, unless it is 'Arvit on Rosh \d{H}odesh.  If the omission is realized before the Amidah is complete, return to Retzei and continue from there (ibid citing OC 422).

Al Hanisim is recited on \d{H}anukkah and Purim.  If omitted the Amidah is not repeated (ibid citing OC 682).

If Aneinu, recited on fast days, is forgotten, the Amidah is not repeated (ibid citing OC 565).

During the Ten Days of Repentence, if "Ha'el HaKadosh" is forgotten and the error is not fixed within a moment, return to the beginning of the Amidah.  Other additions to not require repeating the Amidah.

If a holiday addition was mistakenly recited on a weekday, return to the beginning of the berakha if the berakha has not been concluded.  If the berakha has been completed, continue as usual without returning or repeating.

\subsection{Cases of Doubt}

If one is unsure whether they included a special insertion to the Amidah that is required, they must assume they omitted it and repeat the Amidah.  The same is true for the requests for rain (or lack thereof) within 30 days of beginning / ending including the request.  In such a case, one should have in mind that if they did recite the correct text, the repeated text should be considered a voluntary extra prayer (PH Prayer 18:6).

%\section{Priority List for Aliyot}
%
%A Kohen recieves the first aliya, and a Levi the second.  If there is no Kohen, a Levi need not be called first (or second).  If there is no Levi, the same Kohen is called for the first two aliyot.  In a case of great need the Kohen may be asked to leave so a Yisrael can be called instead.
%
%While there are differences in customs, the following is the priority list for an aliya (based on Artscroll, laws 99-101):\begin{enumerate}
%	\item A person getting married, whether that day (on weekday) or the following week (on Shabbat)
%	\item A child reaching Bar-Mitzva, either on a weekday or on the following Shabbat
%	\item A new parent
%	\item On Shabbat, a person who got married the previous week on Wednesday or later
%	\item Someone observing a Yortzeit
%	\item On Shabbat, a parent whose son will get a Brit Mila in the following week
%	\item Someone who needs to recite Gomel or is leaving on a journey
%\end{enumerate}
%
%A person may waive their precedence to recieve an aliya.
%
%\section{Priority List for the Amud}
%
%On weekdays that are not festive days, it is customary to give mourners priority to lead services.  In former times this was also the priority list to recite kaddish, but nowadays in most commnuities kaddish is recited by mourners collectively.\begin{enumerate}
%	\item 
%\end{enumerate}

\section{Problems in Sifrei Torah}

When a problem in a Sefer Torah is discovered, if it is in a place where it is impossible to end the aliya (less than 3 verses from a paragraph break), a kosher Torah is taken out and the reading is resumed. On Shabbat, if possible, Hosafot should be added to make the remaining reading into 7 aliyot.  If the error occurred in the middle of a sentence, the reading in the kosher Torah should resume from the beginning of that sentence.

If the error was found in a second Torah scroll, the first should not be used.  Instead a third Torah should be used, if available (Applbaum pp 32).

The following are considered errors in the Sefer Torah (Applbaum 32-35):
\begin{itemize}
	\item A missing letter that changes the meaning of the word, including grammatical gender. A change of spelling that does not affect the meaning does not invalidate the Torah, though it should be fixed
	\item One letter split such that it looks like two letters
	\item Two letters joined together to look like one letter
	\item One letter substituted for another
	\item An extra letter
	\item An incorrect paragraph division, either missing or superfluous
	\item The majority of the seam between two sheets is torn (though if no other Torah is available it may be used if the tear is in a different book of the Torah.  If it is in that book, it may be used if there is no other Torah if five seams remain)
	\item Wax or a similar substance obscures the words of the portion being read
\end{itemize}

In a case of doubt, a child is shown the doubtful letter.  If they read it correctly, the Torah may be used, if not it is invalid.

\end{document}